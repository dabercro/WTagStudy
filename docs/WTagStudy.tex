\documentclass{article}

\usepackage[top=1in,bottom=1in,left=1in,right=1in]{geometry}
\usepackage{siunitx}
\usepackage{graphicx}

\graphicspath{{figs/}}

\begin{document}

\section{Hadronic Vector Boson Tagging Scale Factors} \label{sec:wscale}

The goal of this study was to determine the scale factors that should
be applied to MC when tagging for hadronic vector bosons is applied.
We did this by studying hadronic W bosons in a tight $t\bar{t}$ selection.
The result has larger uncertainties due to relatively low yields,
but this will be decreased significantly as more data is gathered.

\subsection{Selection for $t\bar{t}$} \label{sec:wscale_selection}

To match the selection used in the Mono-V category of the analysis,
all events require one anti-$k_T$, $R = 0.8$ ``fat'' jet with $p_T > \SI{250}{GeV}$.
To select semi-leptonic $t\bar{t}$ events, two loose b-tagged anti-$k_T$, $R = 0.4$
jets with $csv > 0.605$ and one tight muon (electron) with
$p_T > \SI{20}{GeV} (\SI{40}{GeV})$ are also required.
Finally, we only allow for one additional anti-$k_T$, $R = 0.4$ jet from ISR, \emph{etc.}
in the event to ensure a clean $t\bar{t}$ selection.
If the other jet is boosted enough to create another fat jet, we use the requirement 
of all other fat jets that is not the selection jet has $m_\text{pruned} < \SI{50}{GeV}$.
Note that this is not a direct cut on the jet being selected.

To evaluate the combinatoric backgrounds, we split the TTJets sample into three regions
as shown in Figure~\ref{fig:wscale_DRGen}.
\begin{itemize}
\item W-matched: in these events, the generator W is within $\Delta R < 0.2$ from the
  reconstructed fat jet. These are our signal events.
\item Pulled off W: In these events, the jet is pulled away from initial W direction.
  From Figure~\ref{fig:wscale_DRb}, we can guess that this is most likely from the jet from
  the W is merged with the b jet from the top decay.
  This is not considered signal.
\item Recoiling Jet: For jets very far from the nearest generator W, we assume the selected
  jet is actually a jet from an additional radiated gluon or pileup.
  All events from the di-lepton $t\bar{t}$ sample is also considered to be from recoiling jets
  since there is no true V-jet to identify in the event.
\end{itemize}

\begin{figure}[h]
  \centering
  \includegraphics[width=0.5\linewidth]{semilep_nocut_fatjetDRGenW.pdf}
  \caption{The TTJets sample is mostly split based on the distance of the reconstructed fat jet
  from the nearest generator W. Jets close to the W are considered to be signal.
  All di-lepton $t\bar{t}$ MC is considered recoiling jets, despite their distance from generator W bosons.}
  \label{fig:wscale_DRGen}
\end{figure}

Additional cuts from the obvious Mono-V and $t\bar{t}$ selection are 
justified by Figures~\ref{fig:wscale_DRb} and \ref{fig:wscale_DPhiLep}.
We use the selection $0.8 < \Delta R(j_b,j_\text{fat}) < 1.2$ 
and $\Delta\phi(\ell,j_\text{fat})$ to cut out the two combinatoric backgrounds.

\begin{figure}[h]
  \centering
  \begin{minipage}{0.49\linewidth}
    \centering
    \includegraphics[width=\linewidth]{semilep_nocut_fatjetDRLooseB.pdf}
    \caption{$\Delta R$ between the reconstructed fat jet and the closest loose
    b-tagged jet is shown. When too close to the b-jet, the fat jet is pulled from
    the W. Recoiling jets are more likely to be far away.}
    \label{fig:wscale_DRb}
  \end{minipage}
  ~
  \begin{minipage}{0.49\linewidth}
    \centering
    \includegraphics[width=\linewidth]{semilep_nocut_fatjetDPhiLep1.pdf}
    \caption{$\Delta\phi$ between the reconstructed fat jet and the tight lepton
      is shown. While not as discriminating as $\Delta R$, this can clean up the selection
      as well.}
    \label{fig:wscale_DPhiLep}
  \end{minipage}
\end{figure}
  
\subsection{Vector Tagging Variables and Scale Factor} \label{sec:wscale_tagging}

We use the loose cut-based tagger of $\SI{65}{GeV} < m_\text{pruned} < \SI{115}{GeV}$
and $\tau_2/\tau_1 < 0.6$.
The distributions of the two variables after the selection described
in the previous section is applied are shown in
Figures~\ref{fig:wscale_massp} and \ref{fig:wscale_tau21}.

\begin{figure}[h]
  \centering
  \begin{minipage}{0.49\linewidth}
    \centering
    \includegraphics[width=\linewidth]{semilep_full_fatjetPrunedM.pdf}
    \caption{The distribution of the pruned mass is shown above.}
    \label{fig:wscale_massp}
  \end{minipage}
  ~
  \begin{minipage}{0.49\linewidth}
    \centering
    \includegraphics[width=\linewidth]{semilep_full_fatjettau21.pdf}
    \caption{The distribution of $\tau_2/\tau_1$ is shown above.}
    \label{fig:wscale_tau21}
  \end{minipage}
\end{figure}
  
The scale factor is determined by counting the W-matched MC and the background-subtracted data.
The MC is normalized to the data before subtraction.
Since there is good agreement between the MC and data, the only uncertainties given in these tables 
are the statistical uncertainties.
Table~\ref{tab:wscale_res} shows resulting scale factors from this.

\begin{table}[h]
  \caption{Scale factors from using all of the data from the previous selection are given.}
  \begin{tabular}{c | c | c | c | c}
    \hline
    & No Cut & Pruned Mass Cut & $\tau_2/\tau_1$ Cut & Full V-tag Cut \\
    \hline
    Background Subtracted Data & 130.97 \pm 13.32 & 85.47 \pm 9.92 & 103.91 \pm 11.07 & 80.67 \pm 9.54 \\
    Truth-matched MC & 143.92 \pm 2.28 & 103.03 \pm 1.93 & 114.09 \pm 2.03 & 97.97 \pm 1.88 \\
    \hline
    Normalized Ratio & 1.00 \pm 0.10 & 0.91 \pm 0.11 & 1.00 \pm 0.11 & 0.90 \pm 0.11 \\
    \hline
  \end{tabular}
  \label{tab:wscale_res}
\end{table}

Historically, the low pruned mass peak is ignored in scale factor measurements. 
In our case, that would significantly reduce the effect of background subtraction needed.
Table~\ref{tab:wscale_hmm} shows the results from this approach.

\begin{table}[h]
  \caption{Scale factors from adding a $m_\text{pruned} > \SI{25}{GeV}$ cut.}
  \begin{tabular}{c | c | c | c | c}
    \hline
    & No Cut & Pruned Mass Cut & $\tau_2/\tau_1$ Cut & Full V-tag Cut \\
    \hline
    Background Subtracted Data & 112.86 \pm 11.89 & 85.63 \pm 9.91 & 104.62 \pm 11.07 & 80.80 \pm 9.54 \\
    Truth-matched MC & 125.79 \pm 2.13 & 103.03 \pm 1.93 & 113.03 \pm 2.02 & 97.97 \pm 1.88 \\
    \hline
    Normalized Ratio & 1.00 \pm 0.11 & 0.93 \pm 0.11 & 1.03 \pm 0.11 & 0.92 \pm 0.11 \\
    \hline
  \end{tabular}
  \label{tab:wscale_hmm}
\end{table}

The two methods of obtaining scale factors are consistent with each other.
  
\end{document}
