\documentclass{article}

\usepackage[top=1in,bottom=1in,left=1in,right=1in]{geometry}
\usepackage{siunitx}
\usepackage{graphicx}

\graphicspath{{figs/}}

\begin{document}

\section{Hadronic Vector Boson Tagging Scale Factors} \label{sec:wscale}

The goal of this study was to determine the scale factors that should
be applied to MC when tagging for hadronic vector bosons is applied.
We did this by studying hadronic W bosons in a tight $t\bar{t}$ selection.
The result has larger uncertainties due to relatively low yields,
but this will be decreased significantly as more data is gathered.

\subsection{Selection for $t\bar{t}$} \label{sec:wscale_selection}

To match the selection used in the Mono-V category of the analysis,
all events require one anti-$k_T$, $R = 0.8$ ``fat'' jet with $p_T > \SI{250}{GeV}$.
To select semi-leptonic $t\bar{t}$ events, two b-tagged anti-$k_T$, $R = 0.4$ jets,
at least one with $csv > 0.89$ (medium working point)
and the other with $csv > 0.605$ (loose working point),
and one tight muon (electron) with
$p_T > \SI{20}{GeV} (\SI{40}{GeV})$ are also required.
Finally, we only allow for one additional anti-$k_T$, $R = 0.4$ jet from ISR, \emph{etc.}
That is, there can only be three jets outside of the fat jet, the other two being
the b-tagged jets.
in the event to ensure a clean $t\bar{t}$ selection.
If the other jet is boosted enough to create another fat jet, we use the requirement 
of all other fat jets that is not the selection jet has $m_\text{pruned} < \SI{50}{GeV}$.
Note that this is not a direct cut on the jet being selected.

To evaluate the combinatoric backgrounds, we split the semi-leptonic TTJets sample into six regions
based on the relative locations of the reco jet and truth-matched particles.
\begin{itemize}
\item W-matched: In this region, two daughter quarks from the closest W are contained inside the jet
  ($\Delta R < 0.8$). There are no truth-matched b quarks inside.
\item Top-matched: In this region, both daughter quarks from the closest W as well as a b quark are
  inside the fat jet.
\item One W and b: In this region, one daughter quark and a b quark are inside the jet
\item Part of W: Only one quark from the W is inside the jet. No b quarks are present.
\item B Quark: A b quark is inside the jet, but no quark daughters of the closest W are present
\item Recoiling Jet: No quark daughters of the closest W are present, and there are no b quarks in the jet.
  This happens when an ISR jet is picked up instead.
\end{itemize}
The distance between the reco jet and the generated $W$
in these regions is shown in Figure~\ref{fig:wscale_DRGen}.

\begin{figure}[h]
  \centering
  \includegraphics[width=0.5\linewidth]{160714/semilep_nocut_fatjetDRGenW.pdf}
  \caption{The TTJets sample is split based on the distance of the reconstructed fat jet
  from various truth-matched quarks.}
  \label{fig:wscale_DRGen}
\end{figure}

Additional cuts from the obvious Mono-V and $t\bar{t}$ selection are 
justified by Figures~\ref{fig:wscale_DRb}, \ref{fig:wscale_DPhiLep}, and \ref{fig:wscale_njetstot}.
We use the selection $0.8 < \Delta R(j_b,j_\text{fat}) < 1.2$,
$\Delta\phi(\ell,j_\text{fat})$, and
$n_\text{jets,$ak_T$ 0.4} > 3$
to cut out combinatoric backgrounds.
This number of jets cut is in addition to requiring the number of jets outside the fat jet
to be three or less, since this is total number of jets, not just the jets outside
the fat jet.

\begin{figure}[h]
  \centering
  \begin{minipage}{0.49\linewidth}
    \centering
    \includegraphics[width=\linewidth]{160714/semilep_nocut_fatjetDRLooseB.pdf}
    \caption{$\Delta R$ between the reconstructed fat jet and the closest loose
    b-tagged jet is shown. When too close to the b-jet, the fat jet is pulled from
    the W. Recoiling jets are more likely to be far away.}
    \label{fig:wscale_DRb}
  \end{minipage}
  ~
  \begin{minipage}{0.49\linewidth}
    \centering
    \includegraphics[width=\linewidth]{160714/semilep_nocut_fatjetDPhiLep1.pdf}
    \caption{$\Delta\phi$ between the reconstructed fat jet and the tight lepton
      is shown. While not as discriminating as $\Delta R$, this can clean up the selection
      as well.}
    \label{fig:wscale_DPhiLep}
  \end{minipage}
  ~
  \begin{minipage}{0.49\linewidth}
    \centering
    \includegraphics[width=\linewidth]{160714/semilep_full_ntau_mediumB_n_jetsTot.pdf}
    \caption{After cutting on $\Delta R(j_b,j_\text{fat})$, and $\Delta\phi(\ell,j_\text{fat})$,
      the number of $ak_{T}$ 0.4 jets with $p_T > \SI{30}{GeV}$ are shown.
      If at least 4 jets are required, this means there are two b-tags, and two prongs of a
      fat jet, or one merged fat jet and an ISR jet.}
    \label{fig:wscale_njetstot}
  \end{minipage}
\end{figure}
  
\subsection{Vector Tagging Variables and Scale Factor} \label{sec:wscale_tagging}

We use the loose cut-based tagger of $\SI{65}{GeV} < m_\text{pruned} < \SI{105}{GeV}$
and $\tau_2/\tau_1 < 0.6$.
The distributions of the two variables after the selection described
in the previous section is applied are shown in
Figures~\ref{fig:wscale_massp} and \ref{fig:wscale_tau21}.
Each distribution after cutting on the other are shown in
Figures~\ref{fig:wscale_massp_nm1} and \ref{fig:wscale_tau21_nm1}.
  
\begin{figure}[h]
  \centering
  \begin{minipage}{0.49\linewidth}
    \centering
    \includegraphics[width=\linewidth]{160714/semilep_full_ntau_mediumB_ntot_fatjetPrunedML2L3.pdf}
    \caption{The distribution of the pruned mass is shown above.}
    \label{fig:wscale_massp}
  \end{minipage}
  ~
  \begin{minipage}{0.49\linewidth}
    \centering
    \includegraphics[width=\linewidth]{160714/semilep_full_ntau_mediumB_ntot_fatjettau21.pdf}
    \caption{The distribution of $\tau_2/\tau_1$ is shown above.}
    \label{fig:wscale_tau21}
  \end{minipage}
\end{figure}

\begin{figure}[h]
  \centering
  \begin{minipage}{0.49\linewidth}
    \centering
    \includegraphics[width=\linewidth]{160714/semilep_full_ntau_mediumB_ntot_massp_tau21_fatjetPrunedML2L3.pdf}
    \caption{The distribution of the pruned mass after cutting on $\tau_2/\tau_1$ is shown above.}
    \label{fig:wscale_massp_nm1}
  \end{minipage}
  ~
  \begin{minipage}{0.49\linewidth}
    \centering
    \includegraphics[width=\linewidth]{160714/semilep_full_ntau_mediumB_ntot_massp_tau21_fatjettau21.pdf}
    \caption{The distribution of $\tau_2/\tau_1$ after cutting on pruned mass is shown above.}
    \label{fig:wscale_tau21_nm1}
  \end{minipage}
\end{figure}

The scale factor is determined by counting the W-matched MC and the background-subtracted data.
The MC is normalized to the data before subtraction.
The backgrounds subtracted are QCD, $W\rightarrow \ell\nu$, single-top no W, and 
di-lepton $tt$ events.
Other MC events failing to match the W are left in to simulate selection of the wrong jet
for results reported.
It was also confirmed, taking out additional backgrounds does not significantly 
change the scale factors.
Table~\ref{tab:wscale_res} shows resulting scale factors from this.
The only uncertainties given in these tables 
are the statistical uncertainties.

\begin{table}[h]
  \caption{Scale factors from using all of the data from the previous selection are given.}
  \begin{tabular}{c | c | c | c | c}
    \hline
    & No Cut & Pruned Mass Cut & $\tau_2/\tau_1$ Cut & Full V-tag Cut \\
    \hline
    Background Subtracted Data & 619.89 $\pm$ 26.52 & 431.34 $\pm$ 21.27 & 470.20 $\pm$ 22.36 & 393.52 $\pm$ 20.26 \\
    Signal-matched MC & 717.49 $\pm$ 9.36 & 488.83 $\pm$ 7.63 & 561.67 $\pm$ 8.11 & 453.18 $\pm$ 7.26 \\
    \hline
    Normalized Ratio & 1.00 $\pm$ 0.04 & 1.02 $\pm$ 0.05 & 0.97 $\pm$ 0.05 & 1.01 $\pm$ 0.05 \\
    \hline
  \end{tabular}
  \label{tab:wscale_res}
\end{table}

Historically, the low pruned mass peak is ignored in scale factor measurements. 
In our case, that would significantly reduce the effect of background subtraction needed.
Table~\ref{tab:wscale_hmm} shows the results from this approach.

\begin{table}[h]
  \caption{Scale factors from adding a $m_\text{pruned} > \SI{25}{GeV}$ cut.}
  \begin{tabular}{c | c | c | c | c}
    \hline
    & No Cut & Pruned Mass Cut & $\tau_2/\tau_1$ Cut & Full V-tag Cut \\
    \hline
    Background Subtracted Data & 542.63 $\pm$ 24.27 & 431.51 $\pm$ 21.26 & 469.28 $\pm$ 22.30 & 393.66 $\pm$ 20.25 \\
    Signal-matched MC & 637.55 $\pm$ 8.76 & 488.83 $\pm$ 7.63 & 556.67 $\pm$ 8.07 & 453.18 $\pm$ 7.26 \\
    \hline
    Normalized Ratio & 1.00 $\pm$ 0.05 & 1.04 $\pm$ 0.05 & 0.99 $\pm$ 0.05 & 1.02 $\pm$ 0.05 \\
    \hline
  \end{tabular}
  \label{tab:wscale_hmm}
\end{table}

The two methods of obtaining scale factors are consistent with each other.

\subsection{Systematics}

\end{document}
