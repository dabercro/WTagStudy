\documentclass{beamer}

\author[D. Abercrombie]{
  Daniel Abercrombie
}

\title{\bf \sffamily First Pass at W-Tag Scale Factors}
\date{\today}

\usecolortheme{dove}

\usepackage[absolute,overlay]{textpos}
\usefonttheme{serif}
\usepackage{appendixnumberbeamer}
\usepackage{isotope}
\usepackage{hyperref}
\usepackage[english]{babel}
\usepackage{amsmath}
\setbeamerfont{frametitle}{size=\Large,series=\bf\sffamily}
\setbeamertemplate{frametitle}[default][center]
\usepackage{siunitx}
\usepackage{tabularx}
\usepackage{makecell}

\setbeamertemplate{navigation symbols}{}
\usepackage{graphicx}
\usepackage{color}
\setbeamertemplate{footline}[text line]{\parbox{1.083\linewidth}{\footnotesize \hfill \insertshortauthor \hfill \insertpagenumber /\inserttotalframenumber}}
\setbeamertemplate{headline}[text line]{\parbox{1.083\linewidth}{\footnotesize \hspace{-0.083\linewidth} \textcolor{blue}{\sffamily \insertsection \hfill \insertsubsection}}}

\usepackage{changepage}

\newcommand{\beginbackup}{
  \newcounter{framenumbervorappendix}
  \setcounter{framenumbervorappendix}{\value{framenumber}}
}
\newcommand{\backupend}{
  \addtocounter{framenumbervorappendix}{-\value{framenumber}}
  \addtocounter{framenumber}{\value{framenumbervorappendix}} 
}

\graphicspath{{figs/}}

\logo{\includegraphics[height=0.35in]{../MIT-logo.pdf}}

\begin{document}

\begin{frame}[nonumbering]
  \titlepage
\end{frame}

\begin{frame}
  \frametitle{Introduction}
  The goal of this study was to determine the scale factors that should
  be applied to MC when tagging for hadronic vector bosons is applied.
  We did this by studying hadronic W bosons in a tight $t\bar{t}$ selection.
  The result has larger uncertainties due to relatively low yields,
  but this will be decreased significantly as more data is gathered.
\end{frame}

\begin{frame}
  \frametitle{Selection}
  \begin{itemize}
  \item one anti-$k_T$, $R = 0.8$ ``fat'' jet with $p_T > \SI{250}{GeV}$ \\
    (to match Mono-V analysis)
  \item two loose b-tagged anti-$k_T$, $R = 0.4$ jets with $csv > 0.605$ and 
    one tight muon (electron) with $p_T > \SI{20}{GeV} (\SI{40}{GeV})$ \\
    (to selection $t\bar{t}$)
  \item allow only one additional anti-$k_T$, $R = 0.4$ jet from ISR, \emph{etc.}
    (clean $t\bar{t}$ selection)
  \item all other fat jets that is not the selection jet has $m_\text{pruned} < \SI{50}{GeV}$
    (to reduce confusion between fat jets)
  \end{itemize}
\end{frame}

\begin{frame}
  \frametitle{Classifying MC Events}
  Classify TTJets events based off distance between the reconstructed jet and the
  generator W boson
  \begin{columns}
    \begin{column}{0.5\linewidth}
      \centering
      \vspace{20pt}
      \includegraphics[width=\linewidth]{semilep_nocut_fatjetDRGenW.pdf}
    \end{column}
    \begin{column}{0.5\linewidth}
      \begin{itemize}
      \item W-matched: These are our signal events.
      \item Pulled off W: most likely from the jet from
        the W is merged with the b jet from the top decay.
      \item Recoiling Jet: we assume the selected
        jet is from an additional radiated gluon or pileup.
      \end{itemize}
    \end{column}
  \end{columns}
\end{frame}

\begin{frame}
  \frametitle{Additional Cuts for Cleaning}
  From looking at these plots, two more cuts were picked for cleaning
  \begin{columns}
    \begin{column}{0.5\linewidth}
      \centering
      \textcolor{blue}{$\Delta R$(fat jet and closest b-tag)}
      \includegraphics[width=\linewidth]{semilep_nocut_fatjetDRLooseB.pdf}
    \end{column}
    \begin{column}{0.5\linewidth}
      \centering
      \textcolor{blue}{$\Delta \phi$(fat jet and lepton)}
      \includegraphics[width=\linewidth]{semilep_nocut_fatjetDPhiLep1.pdf}
    \end{column}
  \end{columns}
  \begin{itemize}
  \item $0.8 < \Delta R(j_b,j_\text{fat}) < 1.2$ 
  \item $\Delta\phi(\ell,j_\text{fat})$
  \end{itemize}
\end{frame}

\begin{frame}
  \frametitle{Vector Tagging Variables}
  After all cuts are applied, $m_\text{pruned}$ and $\tau_2/\tau_1$ look like this
  \vspace{10pt}
  \begin{columns}
    \begin{column}{0.5\linewidth}
      \centering
      \textcolor{blue}{Pruned Mass}
      \includegraphics[width=\linewidth]{semilep_full_fatjetPrunedM.pdf}
    \end{column}
    \begin{column}{0.5\linewidth}
      \centering
      \textcolor{blue}{N-subjettiness}
      \includegraphics[width=\linewidth]{semilep_full_fatjettau21.pdf}
    \end{column}
  \end{columns}
  \vspace{10pt}
  \textcolor{red}{Low mass pruned mass is not well modelled \\ at the moment}
\end{frame}

\begin{frame}
  \frametitle{Scale Factors}
  Apply cuts $\SI{60}{GeV} < m_\text{pruned} < \SI{110}{GeV}$ and $\tau_2/\tau_1 < 0.6$

  Background is subtracted from data with a 100\% uncertainty

  \vspace{10pt}
  {\scriptsize
    \begin{tabular}{c | c | c | c | c}
      \hline
      & No Cut & Pruned Mass Cut & $\tau_2/\tau_1$ cut & Full V-tag cut \\
      \hline
      \makecell{Background \\ Subtracted \\ Data} & 60.33 \pm 23.58 & 36.69 \pm 9.75 & 
      43.69 \pm 14.60 & 35.26 \pm 9.00 \\
      \makecell{W-matched \\ MC} & 55.68 \pm 3.92 & 41.79 \pm 3.36 & 44.21 \pm 3.48 & 39.17 \pm 3.26 \\
      \hline
      \makecell{Normalized \\ Ratio} & 1.00 \pm 0.40 & 0.81 \pm 0.23 & 0.91 \pm 0.31 & 0.83 \pm 0.22 \\
      \hline
    \end{tabular}
  }

  \vspace{10pt}
  This is without ignoring the low pruned mass disagreement (which many people ignore)
\end{frame}

\begin{frame}
  \frametitle{Scale Factors}
  Apply cuts $\SI{60}{GeV} < m_\text{pruned} < \SI{110}{GeV}$ and $\tau_2/\tau_1 < 0.6$

  Background is subtracted from data with a 100\% uncertainty

  \vspace{10pt}
  {\scriptsize
    \begin{tabular}{c | c | c | c | c}
      \hline
      & No Cut & Pruned Mass Cut & $\tau_2/\tau_1$ cut & Full V-tag cut \\
      \hline
      \makecell{Background \\ Subtracted \\ Data} & 49.27 \pm 16.51 & 37.00 \pm 9.42 & 44.61 \pm 13.65 & 
      35.52 \pm 8.73 \\
      \makecell{W-matched \\ MC} & 50.22 \pm 3.69 & 41.79 \pm 3.36 & 43.92 \pm 3.46 & 39.17 \pm 3.26 \\
      \hline
      \makecell{Normalized \\ Ratio} & 1.00 \pm 0.34 & 0.90 \pm 0.24 & 1.04 \pm 0.33 & 0.92 \pm 0.24 \\
      \hline
    \end{tabular}
  }

  \vspace{10pt}
  Cutting on $m_\text{pruned} > \SI{25}{GeV}$ to ignore the region where there is disagreement
  gives a result consistent with 1
\end{frame}

\begin{frame}
  \frametitle{Conclusions}
  \begin{itemize}
  \item Ignoring where the pruning algorithm fails, we get a scale factor close to 1
  \item We should figure out what is happening at low pruned mass
  \item Background subtraction is a large source of uncertainty \\
    better understanding or optimized cuts will lower this
  \item Can also reduce uncertainty using other samples \\
    this was amc@NLO, running over madgraph samples now
  \item Using more data will also decrease uncertainties
  \item We will also create a fully hadronic $t\bar{t}$ selection
  \end{itemize}
\end{frame}

\end{document}
