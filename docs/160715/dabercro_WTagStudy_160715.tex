\documentclass{beamer}

\author[D. Abercrombie]{
  Daniel Abercrombie
}

\title{\bf \sffamily W Tag Update}
\date{\today}

\usecolortheme{dove}

\usepackage[absolute,overlay]{textpos}
\usefonttheme{serif}
\usepackage{appendixnumberbeamer}
\usepackage{isotope}
\usepackage{hyperref}
\usepackage[english]{babel}
\usepackage{amsmath}
\setbeamerfont{frametitle}{size=\Large,series=\bf\sffamily}
\setbeamertemplate{frametitle}[default][center]
\usepackage{siunitx}
\usepackage{tabularx}
\usepackage{makecell}

\setbeamertemplate{navigation symbols}{}
\usepackage{graphicx}
\usepackage{color}
\setbeamertemplate{footline}[text line]{\parbox{1.083\linewidth}{\footnotesize \hfill \insertshortauthor \hfill \insertpagenumber /\inserttotalframenumber}}
\setbeamertemplate{headline}[text line]{\parbox{1.083\linewidth}{\footnotesize \hspace{-0.083\linewidth} \textcolor{blue}{\sffamily \insertsection \hfill \insertsubsection}}}

\usepackage{changepage}

\newcommand{\beginbackup}{
  \newcounter{framenumbervorappendix}
  \setcounter{framenumbervorappendix}{\value{framenumber}}
}
\newcommand{\backupend}{
  \addtocounter{framenumbervorappendix}{-\value{framenumber}}
  \addtocounter{framenumber}{\value{framenumbervorappendix}} 
}

\usepackage{changepage}

\graphicspath{{figs/}}

\begin{document}

\begin{frame}[nonumbering]
  \titlepage
\end{frame}

\begin{frame}
  \frametitle{Introduction}
  \begin{itemize}
  \item \emph{Improvements}: \\
    This is a report of changes in the W-tag scale factor measurement.
    The integrated luminosity has been increased from 2.6 to 7.6\si{fb^{-1}}
    and due to larger yeilds, we made slightly purer cuts.
  \item \emph{Bug-fixes}: \\
    Was using fully corrected $p_T$ ratio to adjust pruned mass, when only
    L2L3 should be used. This makes a significant difference.
  \item \emph{Results}: \\
    The resulting scale factor is now more consistent with unity than
    what we were using before.
%    Including systematics shows these results are not contrary.
    Herwig still disagrees.
  \end{itemize}
\end{frame}

\begin{frame}
  \frametitle{Increase Luminosity}
  Increased from 2.6 to 7.6\si{fb^{-1}}
  \vspace{12pt}
  \begin{columns}
    \begin{column}{0.5\linewidth}
      \includegraphics[width=\linewidth]{160714_lowlumi/semilep_full_fatjetPrunedM.pdf}
    \end{column}
    \begin{column}{0.5\linewidth}
      \includegraphics[width=\linewidth]{160714_oldPU/semilep_full_fatjetPrunedM.pdf}
    \end{column}
  \end{columns}
\end{frame}

\begin{frame}
  \frametitle{Fix Pileup Reweighting}
  Fixed the PU reweighting
  \vspace{12pt}
  \begin{columns}
    \begin{column}{0.5\linewidth}
      \centering
      \textcolor{blue}{PU weight from \SI{2.6}{fb^{-1}}}
      \includegraphics[width=\linewidth]{160714_oldPU/semilep_nocut_npv.pdf}
    \end{column}
    \begin{column}{0.5\linewidth}
      \centering
      \textcolor{blue}{PU weight from \SI{7.6}{fb^{-1}}}
      \includegraphics[width=\linewidth]{160714/semilep_nocut_npv.pdf}
    \end{column}
  \end{columns}
\end{frame}

\begin{frame}
  \frametitle{Corrected Mass}
  Pruning removes pileup, so we should use L2L3 corrections \\
  Some example event dumps in data \\ (row, npv, pt, prunedm uncorr, fully corr, L2L3): \\
  \centering
  \includegraphics[width=0.7\linewidth]{massdump.png}
\end{frame}

\begin{frame}
  \frametitle{Corrected Mass}
  \[
  m_{pruned,corr} = \frac{p_{T,corr}}{p_{T,raw}} \times m_{pruned,raw}
  \]
  \vspace{12pt}
  \begin{columns}
    \begin{column}{0.5\linewidth}
      \centering
      \textcolor{blue}{Fully corrected $p_T$ ratio}
      \includegraphics[width=\linewidth]{160714/semilep_full_fatjetPrunedM.pdf}
    \end{column}
    \begin{column}{0.5\linewidth}
      \centering
      \textcolor{blue}{Only L2L3 corrections}
      \includegraphics[width=\linewidth]{160714/semilep_full_fatjetPrunedML2L3.pdf}
    \end{column}
  \end{columns}
\end{frame}

\begin{frame}
  \frametitle{More Purity Cuts}
  With more data, we can cut on \\ medium $b$ jets and number of total jets
  \vspace{6pt}
  \begin{columns}
    \begin{column}{0.5\linewidth}
      \centering
      \textcolor{blue}{Number of medium $b$ jets}
      \includegraphics[width=\linewidth]{160714/semilep_full_n_bjetsMedium.pdf}
    \end{column}
    \begin{column}{0.5\linewidth}
      \centering
      \textcolor{blue}{Number of total jets $p_T > \SI{30}{GeV}$}
      \includegraphics[width=\linewidth]{160714/semilep_full_n_jetsTot.pdf}
    \end{column}
  \end{columns}
  \vspace{6pt}
  Require at least one medium $b$ and at least 4 total jets
\end{frame}

\begin{frame}
  \frametitle{Background Subtraction}
  We subtract QCD, Wln, ST-t, TTJets\_Di-Lept backgrounds
  \vspace{4pt}
  \begin{columns}
    \begin{column}{0.5\linewidth}
      \centering
      \textcolor{blue}{Before pure cuts}
      \includegraphics[width=0.6\linewidth]
                      {160714_background/semilep_full_fatjetPrunedML2L3.pdf} \\
      \includegraphics[width=0.6\linewidth]
                      {160714_background/semilep_full_fatjettau21.pdf}
    \end{column}
    \begin{column}{0.5\linewidth}
      \centering
      \textcolor{blue}{After pure cuts}
      \includegraphics[width=0.6\linewidth]
                      {160714_background/semilep_full_ntau_mediumB_ntot_fatjetPrunedML2L3.pdf} \\
      \includegraphics[width=0.6\linewidth]
                      {160714_background/semilep_full_ntau_mediumB_ntot_fatjettau21.pdf}
    \end{column}
  \end{columns}
  Incorrect reconstruction and recoiling jets are left since they simulate
  poor jet selection in the signal region
\end{frame}

\begin{frame}
  \frametitle{N minus 1 plots}
  We subtract QCD, Wln, ST-t, TTJets\_Di-Lept backgrounds
  \vspace{4pt}
  \begin{columns}
    \begin{column}{0.5\linewidth}
      \centering
      \includegraphics[width=\linewidth]
                      {160714_background/semilep_full_ntau_mediumB_massp_tau21_fatjetPrunedML2L3.pdf}
    \end{column}
    \begin{column}{0.5\linewidth}
      \centering
      \includegraphics[width=\linewidth]
                      {160714_background/semilep_full_ntau_mediumB_massp_tau21_fatjettau21.pdf}
    \end{column}
  \end{columns}
  Incorrect reconstruction and recoiling jets are left since they simulate
  poor jet selection in the signal region
\end{frame}

\begin{frame}
  \frametitle{Old Scale Factors}

  \begin{adjustwidth}{-1.5em}{-1.5em}
    \centering
    \vspace{6pt}
    \textcolor{blue}{All mass}
    \vspace{6pt}

    {\scriptsize
      \begin{tabular}{c | c | c | c | c}
        \hline
        & No Cut & Pruned Mass Cut & $\tau_2/\tau_1$ Cut & Full V-tag Cut \\
        \hline
        \makecell{Background \\ Subtracted \\ Data} & 289.23 $\pm$ 21.12 & 185.72 $\pm$ 15.22 & 204.80 $\pm$ 16.58 & 172.59 $\pm$ 14.47 \\
        \makecell{W-matched \\ MC} & 257.37 $\pm$ 3.17 & 180.79 $\pm$ 2.66 & 203.14 $\pm$ 2.82 & 171.06 $\pm$ 2.59 \\
        \hline
        \makecell{Normalized \\ Ratio} & 1.00 $\pm$ 0.07 & 0.91 $\pm$ 0.08 & 0.90 $\pm$ 0.07 & 0.90 $\pm$ 0.08 \\
        \hline
      \end{tabular}
    }
    
    \vspace{6pt}
    \textcolor{blue}{$m_\text{pruned} > \SI{25}{GeV}$}
    \vspace{6pt}
    
    {\scriptsize
      \begin{tabular}{c | c | c | c | c}
        \hline
        & No Cut & Pruned Mass Cut & $\tau_2/\tau_1$ Cut & Full V-tag Cut \\
        \hline
        \makecell{Background \\ Subtracted \\ Data} & 237.11 $\pm$ 18.22 & 188.24 $\pm$ 15.19 & 206.96 $\pm$ 16.44 & 174.59 $\pm$ 14.44 \\
        \makecell{W-matched \\ MC} & 225.81 $\pm$ 2.97 & 180.79 $\pm$ 2.66 & 200.67 $\pm$ 2.80 & 171.06 $\pm$ 2.59 \\
        \hline
        \makecell{Normalized \\ Ratio} & 1.00 $\pm$ 0.08 & 0.99 $\pm$ 0.08 & 0.98 $\pm$ 0.08 & 0.97 $\pm$ 0.08 \\
        \hline
      \end{tabular}
    }
  \end{adjustwidth}
\end{frame}

\begin{frame}
  \frametitle{New Scale Factors}
  \begin{adjustwidth}{-1.5em}{-1.5em}
    \centering
    \vspace{6pt}
    \textcolor{blue}{All mass}
    \vspace{6pt}

    {\scriptsize
      \begin{tabular}{c | c | c | c | c}
        \hline
        & No Cut & Pruned Mass Cut & $\tau_2/\tau_1$ Cut & Full V-tag Cut \\
        \hline
        \makecell{Background \\ Subtracted \\ Data} & 619.89 $\pm$ 26.52 & 431.34 $\pm$ 21.27 & 470.20 $\pm$ 22.36 & 393.52 $\pm$ 20.26 \\
        \makecell{Signal-\\ matched MC} & 717.49 $\pm$ 9.36 & 488.83 $\pm$ 7.63 & 561.67 $\pm$ 8.11 & 453.18 $\pm$ 7.26 \\
        \hline
        \makecell{Normalized \\ Ratio} & 1.00 $\pm$ 0.04 & 1.02 $\pm$ 0.05 & 0.97 $\pm$ 0.05 & 1.01 $\pm$ 0.05 \\
        \hline
    \end{tabular}
    }

    \vspace{6pt}
    \textcolor{blue}{$m_\text{pruned} > \SI{25}{GeV}$}
    \vspace{6pt}

    {\scriptsize
      \begin{tabular}{c | c | c | c | c}
        \hline
        & No Cut & Pruned Mass Cut & $\tau_2/\tau_1$ Cut & Full V-tag Cut \\
        \hline
        \makecell{Background \\ Subtracted \\ Data} & 542.63 $\pm$ 24.27 & 431.51 $\pm$ 21.26 & 469.28 $\pm$ 22.30 & 393.66 $\pm$ 20.25 \\
        \makecell{Signal-\\ matched MC} & 637.55 $\pm$ 8.76 & 488.83 $\pm$ 7.63 & 556.67 $\pm$ 8.07 & 453.18 $\pm$ 7.26 \\
        \hline
        \makecell{Normalized \\ Ratio} & 1.00 $\pm$ 0.05 & 1.04 $\pm$ 0.05 & 0.99 $\pm$ 0.05 & 1.02 $\pm$ 0.05 \\
        \hline
      \end{tabular}
    }
  \end{adjustwidth}
\end{frame}

\begin{frame}
  \frametitle{Systematics}
  Coming soon. Need to be careful that I'm applying the right smearing cuts.
\end{frame}

\beginbackup

\begin{frame}
  \frametitle{Backup Slides}
\end{frame}

\begin{frame}
  \frametitle{$tt$ Samples Used}
  \begin{itemize}
  \item Madgraph Samples:
    \begin{itemize}
    \item TTJets\_SingleLeptFromT\_TuneCUETP8M1\_13TeV-madgraphMLM-pythia8
    \item TTJets\_SingleLeptFromTbar\_TuneCUETP8M1\_13TeV-madgraphMLM-pythia8
    \item TTJets\_DiLept\_TuneCUETP8M1\_13TeV-madgraphMLM-pythia8
    \item TTJets\_DiLept\_TuneCUETP8M1\_13TeV-madgraphMLM-pythia8
    \end{itemize}
  \item TT\_TuneEE5C\_13TeV-powheg-herwigpp
  \end{itemize}
\end{frame}

\begin{frame}
  \frametitle{Background Samples Used}
  \begin{itemize}
  \item QCD\_HT*\_TuneCUETP8M1\_13TeV-madgraphMLM-pythia8
  \item ST\_t(W)*13TeV-powhegV2-madspin-pythia8\_TuneCUETP8M1
  \item WW(WZ)(ZZ)\_TuneCUETP8M1\_13TeV-pythia8
  \item WJetsToLNu\_Pt-*\_TuneCUETP8M1\_13TeV-amcatnloFXFX-pythia8
  \end{itemize}
\end{frame}

\begin{frame}
   \frametitle{\small 170124/semilep\_full\_massp\_tau21\_fatjettau21}
   \centering
   \includegraphics[width=0.7\linewidth]{170124/semilep_full_massp_tau21_fatjettau21.pdf}
\end{frame}

\begin{frame}
   \frametitle{\small 170124/semilep\_full\_0\_0\_fatjetPrunedML2L3}
   \centering
   \includegraphics[width=0.7\linewidth]{170124/semilep_full_0_0_fatjetPrunedML2L3.pdf}
\end{frame}

\begin{frame}
   \frametitle{\small 170124/semilep\_full\_massp\_tau21\_fatjetPrunedML2L3}
   \centering
   \includegraphics[width=0.7\linewidth]{170124/semilep_full_massp_tau21_fatjetPrunedML2L3.pdf}
\end{frame}

\begin{frame}
   \frametitle{\small 170124/semilep\_full\_0\_1\_fatjetPrunedML2L3}
   \centering
   \includegraphics[width=0.7\linewidth]{170124/semilep_full_0_1_fatjetPrunedML2L3.pdf}
\end{frame}

\begin{frame}
   \frametitle{\small 170124/semilep\_full\_0\_2\_fatjetPrunedML2L3}
   \centering
   \includegraphics[width=0.7\linewidth]{170124/semilep_full_0_2_fatjetPrunedML2L3.pdf}
\end{frame}

\begin{frame}
   \frametitle{\small 170124/semilep\_full\_0\_3\_fatjetPrunedML2L3}
   \centering
   \includegraphics[width=0.7\linewidth]{170124/semilep_full_0_3_fatjetPrunedML2L3.pdf}
\end{frame}

\begin{frame}
   \frametitle{\small 170124/semilep\_full\_0\_4\_fatjetPrunedML2L3}
   \centering
   \includegraphics[width=0.7\linewidth]{170124/semilep_full_0_4_fatjetPrunedML2L3.pdf}
\end{frame}

\begin{frame}
   \frametitle{\small 170124/semilep\_full\_0\_5\_fatjetPrunedML2L3}
   \centering
   \includegraphics[width=0.7\linewidth]{170124/semilep_full_0_5_fatjetPrunedML2L3.pdf}
\end{frame}

\begin{frame}
   \frametitle{\small 170124/semilep\_full\_0\_6\_fatjetPrunedML2L3}
   \centering
   \includegraphics[width=0.7\linewidth]{170124/semilep_full_0_6_fatjetPrunedML2L3.pdf}
\end{frame}

\begin{frame}
   \frametitle{\small 170124/semilep\_full\_0\_7\_fatjetPrunedML2L3}
   \centering
   \includegraphics[width=0.7\linewidth]{170124/semilep_full_0_7_fatjetPrunedML2L3.pdf}
\end{frame}

\begin{frame}
   \frametitle{\small 170124/semilep\_full\_highpt\_fatjetPrunedML2L3}
   \centering
   \includegraphics[width=0.7\linewidth]{170124/semilep_full_highpt_fatjetPrunedML2L3.pdf}
\end{frame}



\backupend

\end{document}
