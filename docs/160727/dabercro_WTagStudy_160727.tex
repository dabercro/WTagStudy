\documentclass{beamer}

\author[D. Abercrombie]{
  The Monojet Working Group
}

\title{\bf \sffamily V(W)-tagging in the Monojet/Mono-V Analysis}
\date{July 27, 2016}

\usecolortheme{dove}

\usepackage{hyperref}
\usepackage{xcolor}

\usepackage[absolute,overlay]{textpos}
\usefonttheme{serif}
\usepackage{appendixnumberbeamer}
\usepackage{isotope}
\usepackage{hyperref}
\usepackage[english]{babel}
\usepackage{amsmath}
\setbeamerfont{frametitle}{size=\Large,series=\bf\sffamily}
\setbeamertemplate{frametitle}[default][center]
\usepackage{siunitx}
\usepackage{tabularx}
\usepackage{makecell}

\setbeamertemplate{navigation symbols}{}
\usepackage{graphicx}
\usepackage{color}
\setbeamertemplate{footline}[text line]{\parbox{1.083\linewidth}{\footnotesize \hfill \insertshortauthor \hfill \insertpagenumber /\inserttotalframenumber}}
\setbeamertemplate{headline}[text line]{\parbox{1.083\linewidth}{\footnotesize \hspace{-0.083\linewidth} \textcolor{blue}{\sffamily \insertsection \hfill \insertsubsection}}}

\usepackage{changepage}

\newcommand{\beginbackup}{
  \newcounter{framenumbervorappendix}
  \setcounter{framenumbervorappendix}{\value{framenumber}}
}
\newcommand{\backupend}{
  \addtocounter{framenumbervorappendix}{-\value{framenumber}}
  \addtocounter{framenumber}{\value{framenumbervorappendix}} 
}

\graphicspath{{figs/}}

\begin{document}

\begin{frame}[nonumbering]
  \titlepage
\end{frame}

\begin{frame}
  \frametitle{Scale Factor Needed for Mono-V}
  As of the time of this talk, the scale factor section of the analysis note
  has plots with $\SI{7.6}{fb^{-1}}$.
  Measurements with $\SI{12.9}{fb^{-1}}$ were made at the beginning of this week
  and are presented in this talk.

  \begin{center}
    \includegraphics[width=0.4\linewidth]{met_SR.pdf} \\
    \textcolor{blue}{From AN-16-195}
  \end{center}

  The scale factor is for di-boson, top, and signal samples.
\end{frame}

\begin{frame}
  \frametitle{Introduction to Measurement}
  The goal of this study was to determine the scale factors that should
  be applied to MC when tagging for hadronic vector bosons is applied.
  We did this by studying hadronic W bosons in a tight $t\bar{t}$ selection.

  \vspace{12pt}

  Given a tight enough selection and enough data, we can calculate the 
  scale factor with a cut and count method.
  Though conservative systematic uncertainties are applied, this method still has advantages
  over fitting for taggers where backgrounds are shaped.
  For now, this method is only applied to a loose cut-based tagger,
  but we would like to use it in the future for more sophisticated tagging techniques.
\end{frame}

\begin{frame}
  \frametitle{Samples Used}
  {\scriptsize
  \begin{itemize}
  \item TTJets\_SingleLeptFromT\_TuneCUETP8M1\_13TeV-madgraphMLM-pythia8
  \item TTJets\_SingleLeptFromTbar\_TuneCUETP8M1\_13TeV-madgraphMLM-pythia8
  \item TTJets\_DiLept\_TuneCUETP8M1\_13TeV-madgraphMLM-pythia8
  \item QCD\_HT*\_TuneCUETP8M1\_13TeV-madgraphMLM-pythia8
  \item ST\_t(W)*13TeV-powhegV2-madspin-pythia8\_TuneCUETP8M1
  \item WJetsToLNu\_Pt-*\_TuneCUETP8M1\_13TeV-amcatnloFXFX-pythia8
  \item WW(WZ)(ZZ)\_TuneCUETP8M1\_13TeV-pythia8 [$\dagger$]
  \item DYJetsToLL\_M-50\_TuneCUETP8M1\_13TeV-amcatnloFXFX-pythia8 [$\dagger$]
  \end{itemize}
  [$\dagger$] Ended up with negligible yields, not in plot legends
  }
\end{frame}

\begin{frame}
  \frametitle{Base selection}
  In a picture \\
  \vspace{12pt}
  \begin{columns}
    \begin{column}{0.8\linewidth}
      \centering
      \includegraphics[width=\linewidth]{ttselection.pdf}
    \end{column}
    \begin{column}{0.2\linewidth}
      \centering
      \vspace{-60pt}
      \hspace{-72pt}
      {\scriptsize
      \begin{tabular}{|c|c|c|}
        \hline
        $m_\text{p,1}$ & $m_\text{p,2}$ & Select jet \\
        \hline
        \SI{25}{GeV} & -- & 1 \\
        \SI{80}{GeV} & -- & 1 \\
        \SI{25}{GeV} & \SI{25}{GeV} & 1 \\
        \SI{80}{GeV} & \SI{25}{GeV} & 1 \\
        \SI{25}{GeV} & \SI{80}{GeV} & 2 \\
        \SI{80}{GeV} & \SI{80}{GeV} & Neither \\
        \hline
      \end{tabular}
      }
    \end{column}
  \end{columns}
  \vspace{12pt}
  Both b jets are loose and at least one is medium
\end{frame}

\begin{frame}
  \frametitle{Base selection}
  In words
  \begin{itemize}
  \item one anti-$k_T$, $R = 0.8$ ``fat'' jet with $p_T > \SI{250}{GeV}$ \\
    (to match Mono-V analysis)
  \item two b-tagged anti-$k_T$, $R = 0.4$ jets, one medium with
    \\ CSVv2 $> 0.89$ the other loose with CSVv2 $> 0.605$ \\
    (to select $t\bar{t}$)
  \item one tight muon (electron) with $p_T > \SI{20}{GeV} (\SI{40}{GeV})$ \\
    (to select $t\bar{t}$)
  \item all other fat jets that is not the selection jet has $m_\text{pruned} < \SI{50}{GeV}$
    (to reduce confusion between fat jets)
  \item allow only up to one additional anti-$k_T$, $R = 0.4$ or $R = 0.8$ jet from ISR,
    \emph{etc.} \\ (clean $t\bar{t}$ selection, which allows for extra boost from ISR)
  \end{itemize}
\end{frame}

\begin{frame}
  \frametitle{Tracking combinatorics}
  We split $t\bar{t}$ into six categories
  \begin{itemize}
    {\small
  \item W-matched: two daughter quarks from the closest W are contained inside the jet
    ($\Delta R < 0.8$) no truth-matched b quarks
  \item Top-matched: both daughter quarks from the closest W as well as a b quark are inside
    }
  \end{itemize}
  \begin{columns}
    \begin{column}{0.5\linewidth}
      \begin{itemize}
        \vspace{-24pt}
    {\small
      \item One W and b: one daughter quark and a b quark
      \item Part of W: only one quark from the W, no b quarks
      \item B Quark: A b quark is inside the jet, but no daughters of the closest W
      \item Recoiling Jet: No quark daughters of the closest W or b are present,
        likely ISR
        }
      \end{itemize}
    \end{column}
    \begin{column}{0.5\linewidth}
      \centering
      \textcolor{blue}{Base selection, not final}
      \includegraphics[width=\linewidth]
                      {160726/semilep_nocut_nsmalljets_fatjetPrunedML2L3.pdf}
    \end{column}
  \end{columns}
\end{frame}

\begin{frame}
  \frametitle{Additional Purity Cuts}
  Searching for other discriminating variables gives the following
  \vspace{12pt}
  \begin{columns}
    \begin{column}{0.5\linewidth}
      \centering
      \textcolor{blue}{$\Delta R(j_b,j_\text{fat})$ of closest b-tag}
      \includegraphics[width=\linewidth]
                      {160726/semilep_nocut_nsmalljets_fatjetDRLooseB.pdf}
    \end{column}
    \begin{column}{0.5\linewidth}
      \centering
      \textcolor{blue}{$\Delta\phi(\ell,j_\text{fat})$}
      \includegraphics[width=\linewidth]
                      {160726/semilep_nocut_nsmalljets_fatjetDPhiLep1.pdf}
    \end{column}
  \end{columns}
  \vspace{12pt}
  We use purity cuts $0.8 < \Delta R(j_b,j_\text{fat}) < 1.2$ and
  $\Delta\phi(\ell,j_\text{fat}) > 2.0$
\end{frame}

\begin{frame}
  \frametitle{V-tagging variables}
  This is the selection we use to get clean, hadronic Ws without biasing the tagging variables
  \vspace{12pt}
  \begin{columns}
    \begin{column}{0.5\linewidth}
      \centering
      \textcolor{blue}{Pruned Mass}
      \includegraphics[width=\linewidth]
                      {160726/semilep_full_fatjetPrunedML2L3.pdf}
    \end{column}
    \begin{column}{0.5\linewidth}
      \centering
      \textcolor{blue}{N-subjettiness ($\tau_2/\tau_1$)}
      \includegraphics[width=\linewidth]
                      {160726/semilep_full_fatjettau21.pdf}
    \end{column}
  \end{columns}
  Cut on $m_\text{pruned}$ and $\tau_2/\tau_1$ for V-tagging
\end{frame}

\begin{frame}
  \frametitle{Background Subtraction from Data}
  \begin{itemize}
  \item Normalize MC to Data
  \item Only differences in shape affect the SF \vspace{-2pt} \\
    \textcolor{blue}{\scriptsize \hspace{-35pt} Signal: Axial
      $m_\text{med} = \SI{2000}{GeV}$,
      $m_\text{DM} = \SI{50}{GeV}$}
    \begin{columns}
      \begin{column}{0.4\linewidth}
        \centering
        \includegraphics[width=\linewidth]{160719/fatjetPrunedMx.pdf}
      \end{column}
      \begin{column}{0.6\linewidth}
        \vspace{-12pt}
      \item Only subtract MCs without a hadronic W existing
        \vspace{-6pt}
        \begin{itemize}
        \item $W \rightarrow \ell \nu$
        \item QCD
        \item Single top, ST\_t (not ST\_tW)
        \item Di-leptonic $t\bar{t}$
        \end{itemize}
      \end{column}
    \end{columns}
  \item Two reasons to not subtract more
    \begin{itemize}
    \item A non-reducible background in analysis is $t\bar{t}$
    \item Selecting W is not guaranteed in signal samples
    \end{itemize}
  \item Both of these cases would be poorly represented by
    W-matched background subtraction
  \item Other background subtraction is explored in systematic uncertainties, 
    and made no difference to scale factor
  \end{itemize}
\end{frame}

\begin{frame}
  \frametitle{Scale Factors}
  Statistical uncertainties, systematic uncertainties explored next
  \textcolor{red}{\scriptsize
    Pruned Mass Cut: $\SI{65}{GeV} < m_\text{pruned} < \SI{105}{GeV}$ -- 
    $\tau_2/\tau_1$ Cut: $\tau_2/\tau_1 < 0.6$ \\ \vspace{-12pt}
  }
  \begin{adjustwidth}{-1.5em}{-1.5em}
    \centering
    \vspace{6pt}
    \textcolor{blue}{All mass}
    \vspace{6pt}

    {\scriptsize
      \begin{tabular}{| c | c | c | c | c |}
        \hline
        & No Cut & Pruned Mass Cut & $\tau_2/\tau_1$ Cut & Full V-tag Cut \\
        \hline
        \makecell{Background \\ Subtracted \\ Data} & 1200.81 $\pm$ 38.63 & 738.71 $\pm$ 28.79 & 831.73 $\pm$ 30.71 & 673.15 $\pm$ 27.44 \\
        \makecell{Signal-\\ matched MC} & 1375.32 $\pm$ 16.66 & 860.33 $\pm$ 13.02 & 999.57 $\pm$ 13.88 & 793.25 $\pm$ 12.33 \\
        \hline
        \makecell{Normalized \\ Ratio} & 1.00 & 0.98 $\pm$ 0.04 & 0.95 $\pm$ 0.04 & \fcolorbox{red}{yellow}{0.97 $\pm$ 0.04} \\
        \hline
      \end{tabular}
    }

    \vspace{6pt}
    \textcolor{blue}{$m_\text{pruned} > \SI{25}{GeV}$}
    \vspace{6pt}

    {\scriptsize
      \begin{tabular}{| c | c | c | c | c |}
        \hline
        & No Cut & Pruned Mass Cut & $\tau_2/\tau_1$ Cut & Full V-tag Cut \\
        \hline
        \makecell{Background \\ Subtracted \\ Data} & 981.34 $\pm$ 33.79 & 739.47 $\pm$ 28.73 & 827.66 $\pm$ 30.53 & 673.79 $\pm$ 27.38 \\
        \makecell{Signal-\\ matched MC} & 1156.49 $\pm$ 15.16 & 860.33 $\pm$ 13.02 & 988.32 $\pm$ 13.81 & 793.25 $\pm$ 12.33 \\
        \hline
        \makecell{Normalized \\ Ratio} & 1.00 & 1.01 $\pm$ 0.04 & 0.99 $\pm$ 0.04 & 1.00 $\pm$ 0.04 \\
        \hline
      \end{tabular}
    }
  \end{adjustwidth}
\end{frame}

\begin{frame}
  \frametitle{Systematic Uncertainties}
 Four separate effects were investigated for the systematic uncertainties
  \begin{itemize}
  \item Jet Smearing
  \item Background subtraction magnitude
  \item Which background is subtracted
  \item Shower effect on purity cuts
  \end{itemize}
  We  use the full mass spectrum, so compare to
  \fcolorbox{red}{yellow}{0.97 $\pm$ 0.04} \\
  For each slide with scale factor tables, above will compare pruned mass plots with the base
  cuts we use on the left and the MC adjusted for systematics on the right
\end{frame}

\begin{frame}
  \frametitle{Jet Smearing}
  Used the hybrid smearing procedure [*] which changes the MC's mass and $p_T$ with
  ``Smeared Up'' giving the most different SF
  \begin{columns}
    \begin{column}{0.5\linewidth}
      \centering
      \includegraphics[width=0.7\linewidth]
                      {160726_background/semilep_full_fatjetPrunedML2L3.pdf}
    \end{column}
    \begin{column}{0.5\linewidth}
      \centering
      \includegraphics[width=0.7\linewidth]
                      {160726_background/smearedup_mass.pdf}
    \end{column}
  \end{columns}

  \textcolor{red}{\scriptsize
    Pruned Mass Cut: $\SI{65}{GeV} < m_\text{pruned} < \SI{105}{GeV}$ -- 
    $\tau_2/\tau_1$ Cut: $\tau_2/\tau_1 < 0.6$ \\ \vspace{-12pt}
  }
  \begin{adjustwidth}{-1.5em}{-1.5em}
    \centering
    {\scriptsize
      \begin{tabular}{| c | c | c | c | c |}
        \hline
        & No Cut & Pruned Mass Cut & $\tau_2/\tau_1$ Cut & Full V-tag Cut \\
        \hline
        \makecell{Background \\ Subtracted \\ Data} & 1200.81 $\pm$ 38.63 & 738.39 $\pm$ 28.79 & 832.30 $\pm$ 30.71 & 673.37 $\pm$ 27.44 \\
        \makecell{Signal-\\ matched MC} & 1375.32 $\pm$ 16.66 & 809.55 $\pm$ 12.63 & 947.81 $\pm$ 13.51 & 745.50 $\pm$ 11.99 \\
        \hline
        \makecell{Normalized \\ Ratio} & 1.00 & 1.04 $\pm$ 0.04 & 1.01 $\pm$ 0.04 & \fcolorbox{red}{yellow}{1.03 $\pm$ 0.05} \\
        \hline
      \end{tabular}
    }

  \end{adjustwidth}
  {\small [*] 
    \href{https://twiki.cern.ch/twiki/bin/viewauth/CMS/JetResolution#Smearing_procedures}
         {https://twiki.cern.ch/twiki/bin/viewauth/CMS/JetResolution}}
\end{frame}

\begin{frame}
  \frametitle{Background Subtraction}
  Scaling the backgrounds down to half
  \begin{columns}
    \begin{column}{0.5\linewidth}
      \centering
      \includegraphics[width=0.7\linewidth]
                      {160726_background/semilep_full_fatjetPrunedML2L3.pdf}
    \end{column}
    \begin{column}{0.5\linewidth}
      \centering
      \includegraphics[width=0.7\linewidth]
                      {160727_down/semilep_full_fatjetPrunedML2L3.pdf}
    \end{column}
  \end{columns}
      \textcolor{red}{\scriptsize
    Pruned Mass Cut: $\SI{65}{GeV} < m_\text{pruned} < \SI{105}{GeV}$ -- 
    $\tau_2/\tau_1$ Cut: $\tau_2/\tau_1 < 0.6$ \\ \vspace{-12pt}
  }
  \begin{adjustwidth}{-1.5em}{-1.5em}
    \centering
    {\scriptsize
      \begin{tabular}{| c | c | c | c | c |}
        \hline
        & No Cut & Pruned Mass Cut & $\tau_2/\tau_1$ Cut & Full V-tag Cut \\
        \hline
        \makecell{Background \\ Subtracted \\ Data} & 1255.91 $\pm$ 36.83 & 752.35 $\pm$ 27.96 & 851.87 $\pm$ 29.83 & 684.58 $\pm$ 26.65 \\
        \makecell{Signal-\\ matched MC} & 1375.32 $\pm$ 16.66 & 860.33 $\pm$ 13.02 & 999.57 $\pm$ 13.88 & 793.25 $\pm$ 12.33 \\
        \hline
        \makecell{Normalized \\ Ratio} & 1.00 & 0.96 $\pm$ 0.04 & 0.93 $\pm$ 0.04 & \fcolorbox{red}{yellow}{0.95 $\pm$ 0.04} \\
        \hline
      \end{tabular}
    }
  \end{adjustwidth}
\end{frame}

\begin{frame}
  \frametitle{Background Subtraction}
  Scaling the backgrounds up 100\%
  \begin{columns}
    \begin{column}{0.5\linewidth}
      \centering
      \includegraphics[width=0.7\linewidth]
                      {160726_background/semilep_full_fatjetPrunedML2L3.pdf}
    \end{column}
    \begin{column}{0.5\linewidth}
      \centering
      \includegraphics[width=0.7\linewidth]
                      {160726_up/semilep_full_fatjetPrunedML2L3.pdf}
    \end{column}
  \end{columns}
      \textcolor{red}{\scriptsize
    Pruned Mass Cut: $\SI{65}{GeV} < m_\text{pruned} < \SI{105}{GeV}$ -- 
    $\tau_2/\tau_1$ Cut: $\tau_2/\tau_1 < 0.6$ \\ \vspace{-12pt}
  }
  \begin{adjustwidth}{-1.5em}{-1.5em}
    \centering
    {\scriptsize
      \begin{tabular}{| c | c | c | c | c |}
        \hline
        & No Cut & Pruned Mass Cut & $\tau_2/\tau_1$ Cut & Full V-tag Cut \\
        \hline
        \makecell{Background \\ Subtracted \\ Data} & 1090.62 $\pm$ 45.13 & 711.41 $\pm$ 31.90 & 791.46 $\pm$ 34.01 & 650.30 $\pm$ 30.39 \\
        \makecell{Signal-\\ matched MC} & 1375.32 $\pm$ 16.66 & 860.33 $\pm$ 13.02 & 999.57 $\pm$ 13.88 & 793.25 $\pm$ 12.33 \\
        \hline
        \makecell{Normalized \\ Ratio} & 1.00 & 1.04 $\pm$ 0.05 & 1.00 $\pm$ 0.05 & \fcolorbox{red}{yellow}{1.03 $\pm$ 0.05} \\
        \hline
      \end{tabular}
    }
  \end{adjustwidth}
\end{frame}

\begin{frame}
  \frametitle{Background Selection}
  If we require the jet to be matched to the W (both daughters inside, no b)
  \begin{columns}
    \begin{column}{0.5\linewidth}
      \centering
      \includegraphics[width=0.7\linewidth]
                      {160726_background/semilep_full_fatjetPrunedML2L3.pdf}
    \end{column}
    \begin{column}{0.5\linewidth}
      \centering
      \includegraphics[width=0.7\linewidth]
                      {160726_morebackground/semilep_full_fatjetPrunedML2L3.pdf}
    \end{column}
  \end{columns}
      \textcolor{red}{\scriptsize
    Pruned Mass Cut: $\SI{65}{GeV} < m_\text{pruned} < \SI{105}{GeV}$ -- 
    $\tau_2/\tau_1$ Cut: $\tau_2/\tau_1 < 0.6$ \\ \vspace{-12pt}
  }
  \begin{adjustwidth}{-1.5em}{-1.5em}
    \centering
    {\scriptsize
      \begin{tabular}{| c | c | c | c | c |}
        \hline
        & No Cut & Pruned Mass Cut & $\tau_2/\tau_1$ Cut & Full V-tag Cut \\
        \hline
        \makecell{Background \\ Subtracted \\ Data} & 770.38 $\pm$ 39.81 & 615.02 $\pm$ 29.34 & 659.40 $\pm$ 31.33 & 579.16 $\pm$ 27.85 \\
        \makecell{Signal-\\ matched MC} & 882.34 $\pm$ 12.51 & 718.67 $\pm$ 11.30 & 802.20 $\pm$ 11.95 & 685.60 $\pm$ 11.04 \\
        \hline
        \makecell{Normalized \\ Ratio} & 1.00 & 0.98 $\pm$ 0.05 & 0.94 $\pm$ 0.05 & \fcolorbox{red}{yellow}{0.97 $\pm$ 0.05} \\
        \hline
      \end{tabular}
    }
  \end{adjustwidth}
\end{frame}

\begin{frame}
  \frametitle{Shower Effect on $\Delta R(j,b)$ Cuts}
  Scanned over a range of $\Delta R$ cuts
  \begin{columns}
    \begin{column}{0.32\linewidth}
      \centering
      \textcolor{blue}{With tops (0.5, 0.9)} \\
      \includegraphics[width=\linewidth]
                      {160726/semilep_full_0_0_fatjetPrunedML2L3.pdf}
    \end{column}
    \begin{column}{0.32\linewidth}
      \centering
      \textcolor{blue}{Begin scan (0.6, 1.0)} \\
      \includegraphics[width=\linewidth]
                      {160726/semilep_full_0_1_fatjetPrunedML2L3.pdf}
    \end{column}
    \begin{column}{0.32\linewidth}
      \centering
      \textcolor{blue}{Base cut (0.8, 1.2)} \\
      \includegraphics[width=\linewidth]
                      {160726/semilep_full_0_3_fatjetPrunedML2L3.pdf}
    \end{column}
  \end{columns}
  \begin{columns}
    \begin{column}{0.5\linewidth}
      \centering
      \textcolor{blue}{Maximum diff in SF (1.0, 1.4)} \\
      \includegraphics[width=0.7\linewidth]
                      {160726/semilep_full_0_5_fatjetPrunedML2L3.pdf}
    \end{column}
    \begin{column}{0.5\linewidth}
      \centering
      \textcolor{blue}{End of Scan (1.2, 1.6)} \\
      \includegraphics[width=0.7\linewidth]
                      {160726/semilep_full_0_7_fatjetPrunedML2L3.pdf}
    \end{column}
  \end{columns}
\end{frame}

\begin{frame}
  \frametitle{Shower effect}
  Compare $0.8 < \Delta R < 1.2$ to $1.0 < \Delta R < 1.4$
  \begin{columns}
    \begin{column}{0.5\linewidth}
      \centering
      \includegraphics[width=0.7\linewidth]
                      {160726_background/semilep_full_0_3_fatjetPrunedML2L3.pdf}
    \end{column}
    \begin{column}{0.5\linewidth}
      \centering
      \includegraphics[width=0.7\linewidth]
                      {160726_background/semilep_full_0_5_fatjetPrunedML2L3.pdf}
    \end{column}
  \end{columns}
      \textcolor{red}{\scriptsize
    Pruned Mass Cut: $\SI{65}{GeV} < m_\text{pruned} < \SI{105}{GeV}$ -- 
    $\tau_2/\tau_1$ Cut: $\tau_2/\tau_1 < 0.6$ \\ \vspace{-12pt}
  }
  \begin{adjustwidth}{-1.5em}{-1.5em}
    \centering
    {\scriptsize
      \begin{tabular}{| c | c | c | c | c |}
        \hline
        & No Cut & Pruned Mass Cut & $\tau_2/\tau_1$ Cut & Full V-tag Cut \\
        \hline
        \makecell{Background \\ Subtracted \\ Data} & 1075.46 $\pm$ 37.16 & 659.46 $\pm$ 26.35 & 743.84 $\pm$ 29.24 & 602.89 $\pm$ 25.06 \\
        \makecell{Signal-\\ matched MC} & 1219.22 $\pm$ 16.12 & 740.35 $\pm$ 12.31 & 881.44 $\pm$ 13.42 & 689.94 $\pm$ 11.81 \\
        \hline
        \makecell{Normalized \\ Ratio} & 1.00 & 1.01 $\pm$ 0.04 & 0.96 $\pm$ 0.04 & \fcolorbox{red}{yellow}{0.99 $\pm$ 0.04} \\
        \hline
      \end{tabular}
    }
  \end{adjustwidth}
\end{frame}

\begin{frame}
  \frametitle{Systematic Uncertainties}
  We assume systematic uncertainties are symmetric for now
  \begin{center}
  \begin{tabular}{l|c|c}
    Source & All Mass & No Low Mass \\
    \hline
    Jet Smearing & 6\% & 7\% \\
    Background Subtraction & 6\% & 4\% \\
    Which background is subtracted & 0\% & 0\% \\
    Shower effect on purity cuts & 2\% & 1\% \\
  \end{tabular}
  \end{center}
  Adding in Quadrature for all mass: \boxed{$\pm 9\%$} \\
  The dump of SF tables for these studies are in the backup slides \\
  \vspace{12pt}
\end{frame}

\begin{frame}
  \frametitle{Conclusions}
  We suggest using the entire mass spectrum for the W-tagging scale factor.
  This may not be the scale factor for vectors only, but it represents our selection.
  The final scale factor is therefore: \\
  
  \begin{center}
  \boxed{$0.97 \pm 0.04 \text{(stat.)} \pm 0.09 \text{(sys.)}$}
  \end{center}

  This looks pretty close to unity, which is used in the analysis presently.
\end{frame}

\beginbackup

\begin{frame}
  \frametitle{Backup Slides}
\end{frame}

\begin{frame}
  \frametitle{V-tagging variables}
  Each variable is shown after cutting on the other variable
  \vspace{12pt}
  \begin{columns}
    \begin{column}{0.5\linewidth}
      \centering
      \textcolor{blue}{Pruned Mass}
      \includegraphics[width=\linewidth]
                      {160726/semilep_full_massp_tau21_fatjetPrunedML2L3.pdf}
    \end{column}
    \begin{column}{0.5\linewidth}
      \centering
      \textcolor{blue}{N-subjettiness ($\tau_2/\tau_1$)}
      \includegraphics[width=\linewidth]
                      {160726/semilep_full_massp_tau21_fatjettau21.pdf}
    \end{column}
  \end{columns}
\end{frame}

\begin{frame}
  \frametitle{Subtract other background processes \\ (Allow part of W)}
  \begin{adjustwidth}{-1.5em}{-1.5em}
    \centering
    {\scriptsize
      \begin{tabular}{| c | c | c | c | c |}
        \hline
        & No Cut & Pruned Mass Cut & $\tau_2/\tau_1$ Cut & Full V-tag Cut \\
        \hline
        \makecell{Background \\ Subtracted \\ Data} & 1091.25 $\pm$ 39.07 & 683.84 $\pm$ 29.13 & 759.30 $\pm$ 31.04 & 630.14 $\pm$ 27.69 \\
        \makecell{Signal-\\ matched MC} & 1249.84 $\pm$ 15.26 & 797.49 $\pm$ 11.98 & 916.61 $\pm$ 12.89 & 743.99 $\pm$ 11.55 \\
        \hline
        \makecell{Normalized \\ Ratio} & 1.00 & 0.98 $\pm$ 0.04 & 0.95 $\pm$ 0.04 & 0.97 $\pm$ 0.05 \\
        \hline
      \end{tabular}
    }
  \end{adjustwidth}
\end{frame}

\begin{frame}
  \frametitle{Change cut $p_T > \SI{350}{GeV}$ (as opposed to \SI{250}{GeV})}
  \begin{adjustwidth}{-1.5em}{-1.5em}
    \centering
    \vspace{6pt}
    \textcolor{blue}{All mass}
    \vspace{6pt}

    {\scriptsize
      \begin{tabular}{| c | c | c | c | c |}
        \hline
        & No Cut & Pruned Mass Cut & $\tau_2/\tau_1$ Cut & Full V-tag Cut \\
        \hline
        \makecell{Background \\ Subtracted \\ Data} & 230.28 $\pm$ 16.93 & 141.39 $\pm$ 12.67 & 163.45 $\pm$ 14.01 & 122.13 $\pm$ 11.85 \\
        \makecell{Signal-\\ matched MC} & 267.09 $\pm$ 7.02 & 165.72 $\pm$ 5.40 & 195.65 $\pm$ 5.78 & 149.24 $\pm$ 4.96 \\
        \hline
        \makecell{Normalized \\ Ratio} & 1.00 & 0.99 $\pm$ 0.09 & 0.97 $\pm$ 0.09 & 0.95 $\pm$ 0.10 \\
        \hline
      \end{tabular}
    }

    \vspace{6pt}
    \textcolor{blue}{$m_\text{pruned} > \SI{25}{GeV}$}
    \vspace{6pt}

    {\scriptsize
      \begin{tabular}{| c | c | c | c | c |}
        \hline
        & No Cut & Pruned Mass Cut & $\tau_2/\tau_1$ Cut & Full V-tag Cut \\
        \hline
        \makecell{Background \\ Subtracted \\ Data} & 188.94 $\pm$ 15.10 & 141.72 $\pm$ 12.62 & 161.58 $\pm$ 13.83 & 122.47 $\pm$ 11.80 \\
        \makecell{Signal-\\ matched MC} & 230.63 $\pm$ 6.42 & 165.72 $\pm$ 5.40 & 192.03 $\pm$ 5.73 & 149.24 $\pm$ 4.96 \\
        \hline
        \makecell{Normalized \\ Ratio} & 1.00 & 1.04 $\pm$ 0.10 & 1.03 $\pm$ 0.09 & 1.00 $\pm$ 0.10 \\
        \hline
      \end{tabular}
    }
  \end{adjustwidth}
\end{frame}

\begin{frame}
   \frametitle{\small 170124/semilep\_full\_massp\_tau21\_fatjettau21}
   \centering
   \includegraphics[width=0.7\linewidth]{170124/semilep_full_massp_tau21_fatjettau21.pdf}
\end{frame}

\begin{frame}
   \frametitle{\small 170124/semilep\_full\_0\_0\_fatjetPrunedML2L3}
   \centering
   \includegraphics[width=0.7\linewidth]{170124/semilep_full_0_0_fatjetPrunedML2L3.pdf}
\end{frame}

\begin{frame}
   \frametitle{\small 170124/semilep\_full\_massp\_tau21\_fatjetPrunedML2L3}
   \centering
   \includegraphics[width=0.7\linewidth]{170124/semilep_full_massp_tau21_fatjetPrunedML2L3.pdf}
\end{frame}

\begin{frame}
   \frametitle{\small 170124/semilep\_full\_0\_1\_fatjetPrunedML2L3}
   \centering
   \includegraphics[width=0.7\linewidth]{170124/semilep_full_0_1_fatjetPrunedML2L3.pdf}
\end{frame}

\begin{frame}
   \frametitle{\small 170124/semilep\_full\_0\_2\_fatjetPrunedML2L3}
   \centering
   \includegraphics[width=0.7\linewidth]{170124/semilep_full_0_2_fatjetPrunedML2L3.pdf}
\end{frame}

\begin{frame}
   \frametitle{\small 170124/semilep\_full\_0\_3\_fatjetPrunedML2L3}
   \centering
   \includegraphics[width=0.7\linewidth]{170124/semilep_full_0_3_fatjetPrunedML2L3.pdf}
\end{frame}

\begin{frame}
   \frametitle{\small 170124/semilep\_full\_0\_4\_fatjetPrunedML2L3}
   \centering
   \includegraphics[width=0.7\linewidth]{170124/semilep_full_0_4_fatjetPrunedML2L3.pdf}
\end{frame}

\begin{frame}
   \frametitle{\small 170124/semilep\_full\_0\_5\_fatjetPrunedML2L3}
   \centering
   \includegraphics[width=0.7\linewidth]{170124/semilep_full_0_5_fatjetPrunedML2L3.pdf}
\end{frame}

\begin{frame}
   \frametitle{\small 170124/semilep\_full\_0\_6\_fatjetPrunedML2L3}
   \centering
   \includegraphics[width=0.7\linewidth]{170124/semilep_full_0_6_fatjetPrunedML2L3.pdf}
\end{frame}

\begin{frame}
   \frametitle{\small 170124/semilep\_full\_0\_7\_fatjetPrunedML2L3}
   \centering
   \includegraphics[width=0.7\linewidth]{170124/semilep_full_0_7_fatjetPrunedML2L3.pdf}
\end{frame}

\begin{frame}
   \frametitle{\small 170124/semilep\_full\_highpt\_fatjetPrunedML2L3}
   \centering
   \includegraphics[width=0.7\linewidth]{170124/semilep_full_highpt_fatjetPrunedML2L3.pdf}
\end{frame}



\backupend

\end{document}
